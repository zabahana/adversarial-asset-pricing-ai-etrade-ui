\documentclass[11pt,a4paper]{article}

\usepackage[utf8]{inputenc}
\usepackage[T1]{fontenc}
\usepackage{amsmath,amssymb,amsthm}
\usepackage{graphicx}
\usepackage{booktabs}
\usepackage{algorithm}
\usepackage{algorithmic}
\usepackage{hyperref}
\usepackage{natbib}
\usepackage{geometry}
\usepackage{bm}

\geometry{margin=1in}

\title{\textbf{Chapter 1: Introduction and Problem Formulation}\\ARRL Multi-Head Attention Asset Pricing Agent}
\author{ARRL Research Team}
\date{\today}

\begin{document}

\maketitle

\section{Introduction}
\label{sec:intro}

Financial asset pricing represents one of the most challenging problems in quantitative finance, combining high-dimensional feature spaces, temporal dependencies, and adversarial market conditions. This chapter establishes the theoretical and practical foundations for the Multi-Head Attention Deep Q-Network (MHA-DQN) framework designed to address these challenges.

\section{Problem Statement}

\subsection{Asset Pricing as Sequential Decision Making}

Asset pricing can be formulated as a sequential decision-making problem where an agent must decide optimal trading actions based on current market state. The complexity arises from:

\begin{enumerate}
    \item \textbf{High Dimensionality}: Market states incorporate hundreds of features across multiple asset classes
    \item \textbf{Non-Stationarity}: Market dynamics evolve over time
    \item \textbf{Adversarial Environment}: Market conditions can be manipulated
    \item \textbf{Long-Range Dependencies}: Historical events influence future prices
\end{enumerate}

\subsection{Markov Decision Process Formulation}

We model asset pricing as a Markov Decision Process (MDP) $\mathcal{M} = (\mathcal{S}, \mathcal{A}, \mathcal{P}, \mathcal{R}, \gamma)$:

\begin{itemize}
    \item \textbf{State Space} $\mathcal{S} \subset \RR^d$: Feature vectors encoding:
    \begin{itemize}
        \item Price history and technical indicators
        \item Macroeconomic indicators
        \item Cross-asset correlations
        \item Sentiment signals
    \end{itemize}
    
    \item \textbf{Action Space} $\mathcal{A} = \{a_1, a_2, a_3\}$:
    \begin{itemize}
        \item $a_1 = \text{BUY}$: Take long position
        \item $a_2 = \text{HOLD}$: Maintain current position
        \item $a_3 = \text{SELL}$: Close or short position
    \end{itemize}
    
    \item \textbf{Reward Function} $\mathcal{R}: \mathcal{S} \times \mathcal{A} \rightarrow \RR$:
    \begin{equation}
    r_t = \frac{V_t - V_{t-1}}{V_{t-1}} - \lambda \cdot \text{TransactionCost}(a_t)
    \end{equation}
    where $V_t$ is portfolio value and $\lambda$ penalizes transaction costs.
    
    \item \textbf{Discount Factor} $\gamma = 0.99$: Balances immediate vs. future returns
\end{itemize}

\section{Objective Function}

The agent's goal is to maximize expected discounted return:

\begin{equation}
\label{eq:objective}
J(\pi) = \EE_{\pi} \left[ \sum_{t=0}^{\infty} \gamma^t r_t \middle| s_0 \sim \rho_0 \right]
\end{equation}

where $\rho_0$ is the initial state distribution.

\section{Challenges and Contributions}

\subsection{Key Challenges}

\begin{enumerate}
    \item \textbf{Feature Interaction}: Capturing complex relationships among heterogeneous features
    \item \textbf{Temporal Modeling}: Learning long-range dependencies in price sequences
    \item \textbf{Robustness}: Maintaining performance under adversarial conditions
    \item \textbf{Sample Efficiency}: Learning from limited historical data
\end{enumerate}

\subsection{Our Contributions}

\begin{enumerate}
    \item Multi-head attention mechanism for feature interaction modeling
    \item Deep Q-learning for optimal sequential decision-making
    \item Adversarial training framework for robustness
    \item Comprehensive theoretical analysis and convergence guarantees
\end{enumerate}

\section{Document Organization}

This methodology document is organized as follows:

\begin{itemize}
    \item \textbf{Chapter 1} (this chapter): Introduction and problem formulation
    \item \textbf{Chapter 2}: Theoretical foundations and approximation theory
    \item \textbf{Chapter 3}: Multi-head attention mechanism with detailed derivations
    \item \textbf{Chapter 4}: Deep Q-network architecture
    \item \textbf{Chapter 5}: Adversarial robustness framework
    \item \textbf{Chapter 6}: Training methodology
    \item \textbf{Chapter 7}: Evaluation metrics and performance analysis
    \item \textbf{Chapter 8}: Theoretical guarantees and bounds
\end{itemize}

\end{document}

